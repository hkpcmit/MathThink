\documentclass[14pt]{extarticle}
\begin{document}
	\paragraph{Question}
	Say whether the following is true or false and support your answer by a proof: The sum of any five consecutive integers is divisible by 5 (without remainder).
	\paragraph{Claim} The sum of any five consecutive integers is divisible by 5.
	\paragraph{Proof} By algebra and definition of divisibility.
	\par\bigskip
	For any arbitrary consecutive 5 integers, let $n \in \mathcal{Z}$ be the smallest integer among them.  In other words, the 5 consecutive integers are:
	\begin{center} $n, n+1, n+2, n+3, n+4$. 
	\end{center}
	\par
	We shall show that $5|\sum\limits_{i=n}^{n+4} i$ is true or
	\par\begin{equation}\label{eq:1}
	(\exists m \in \mathcal{Z}) (5m = \sum\limits_{i=n}^{n+4} i)
	\end{equation}
	\par\bigskip\begin{center}
	$\sum\limits_{i=n}^{n+4} i= n+n+1+n+2+n+3+n+4$\par
	$=5n+10=5(n+2)$
	\end{center}
	\par\bigskip
	Let $m = n+2$.  Clearly, $m$ is an integer and satisfies (\ref{eq:1}).  Hence by definition of divisibility, $(\forall n \in \mathcal{Z})(5|\sum\limits_{i=n}^{n+4}i)$.
	In other words, the sum of any five consecutive integers is divisible by 5.
	
\end{document}