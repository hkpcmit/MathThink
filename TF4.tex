\documentclass[14pt]{extarticle}
\begin{document}
	\paragraph{Question/Claim}
	Prove that every odd natural number is of one of the forms $4n+1$ or $4n+3$, where n is an integer.
	\paragraph{Proof} By Division Theorem and elimination of invalid forms.
	\par\bigskip
	By the Division Theorem, any integer can be expressed in one of the following forms:
	\begin{equation}\label{eq:1}
	4n, 4n+1, 4n+2, 4n+3, n \in \mathcal{Z}.
	\end{equation}
	Since natural numbers are subset of integers, each of them can also be expressed in one of the forms listed in (\ref{eq:1}).
	\par\bigskip
	Among those listed in (\ref{eq:1}),
	\begin{center}
		$4n = 2(2n)$
	\end{center}
	which is clearly even since $2|4n$ is true by definition of divisibility.  Similarly,
	\begin{center}
		$4n+2 = 2(2n+1)$
	\end{center}
	which is also even since $2|(4n+2)$ is true by definition of divisibility.
	\par\bigskip
	Since both $4n, 4n+2$ are even, odd numbers cannot be expressed as such.  Hence, every odd natural number is of one of the remaining forms $4n+1$ or $4n+3, n \in \mathcal{Z}$.
\end{document}