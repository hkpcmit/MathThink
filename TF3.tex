\documentclass[14pt]{extarticle}
\begin{document}
	\paragraph{Question}
	Say whether the following is true or false and support your answer by a proof: For any integer n, the number $n^2+n+1$ is odd.
	\paragraph{Claim}
	$(\forall n \in \mathcal{Z}) (n^2+n+1)$ is odd.
	\paragraph{Proof} By algebra and definition of divisibility.
	\par\bigskip
	Let arbitrary $n \in \mathcal{Z}$ be given. $(n^2+n+1)$ being odd is equivalent to: 
	\begin{center}
	$(\exists m \in \mathcal{Z}) (2m+1 = n^2+n+1)$
	\par\bigskip
	$(\exists m \in \mathcal{Z}) (2m = n^2+n)$
	\end{center}
	\begin{equation}\label{eq:1}
	(\exists m \in \mathcal{Z}) [2m = n(n+1)]
	\end{equation}
	We shall prove that (\ref{eq:1}) is true by showing $n(n+1)$ is even when $n$ is either even and odd.  
	\paragraph{Case 1}
	Assume $n$ is even.  Therefore, there exists $a \in \mathcal{Z}$ such that $n = 2a$.
	\begin{center}
	$n(n+1) = 2a(2a+1)$
	\end{center}
	Let $m=a(2a+1)$.  Clearly $m$ is an integer and satisfies (\ref{eq:1}).  In other words, $n(n+1)$ is even because it is product of even and any integers.
	\paragraph{Case 2}
	Assume $n$ is odd.  Therefore, there exists $b \in \mathcal{Z}$ such that $n = 2b+1$.
	\begin{center}
	$n(n+1) = (2b+1)(2b+2) = 2(2b+1)(b+1)$		
	\end{center}
	Let $m=(2b+1)(b+1)$.  Clearly, $m$ is an integer and satisfies (\ref{eq:1}).  In other words, $n(n+1)$ is also even because one of its factor: $n+1$ is even.
	\par\bigskip
	Since any integer: $n$ is either even or odd, both cases above show that: $(\forall n \in \mathcal{Z}) (n^2+n)$ is even; and $(\forall n \in \mathcal{Z}) (n^2+n+1)$ is odd.
	
\end{document}